The choice for MPI was justified by the need for industry and the universal aspect of the message passing interface paradigm. For the N-body problem, the messages to be passed are mainly the updated positions and velocities of the bodies which were calculated on the other nodes. An efficient way of storing the data must be conceived to minimize the transfer times and allow the Barnes-Hut algorithm to run. As mentioned previously, two strategies are implemented: a brute-force approach and the Barnes-Hut algorithm.\\
The flow graphs and timing diagrams are very similar for both approaches. Once the initial data has been loaded by the main node, this data is sent to all the workers for them to perform the calculations. As Figure \ref{fig:flow_chart} shows, the data is initially sent using MPI's broadcast and scatter functionalities. Both solutions contain the computation step, but the Barnes-hut algorithm has an extra one, building the quad-tree. At the end of each time step, the new positions and velocities are synchronized across all nodes.\\
\begin{figure}[H]
\centering
\inputTikz{0.95}{../figures/flowchart_bf.tex}
\caption{Flowchart for the brute-force parallel implementation. Once the initial data is sent to all nodes, they update the positions of all bodies which are assigned to them, then send to all of the other nodes the updates positions. The step where the quad-tree is built (C) is only present for Barnes-Hut algorithm, in the brute-force approach this step does no exist.}
\label{fig:flow_chart}
\end{figure}
\subsection{Load-balancing}\label{sec:lb}
As the brute-force strategy is load-balanced by construction, nothing needs to be done to keep each node doing the same work. On the other hand, the quad-tree approach needs to be carefully load-balanced to keep each node doing the same amount of work. Since the quad-tree stores the information based on spatial coordinates, simply splitting the tree at the same depth by distributing to each node a branch is not acceptable. By doing this, the case would often occur that the most of the bodies are found in one quadrant, whilst the other quadrants are empty.\\
To solve this problem, the tree is distributed by performing a tree walk. Starting from the root of the tree, following a depth-first search, each  branch tries to be assigned to a node. For this to happen, the compute-nodes can be imaged as bins with a capacity. As the tree is explored, the branch are either assigned to a compute-node if the capacity allows it, or the branch is split into its sub-branches and the process is started again. The code shown in Figure \ref{fig:load-balancing} presents this idea.
\begin{figure}[H]
  \begin{adjustwidth}{-0.5cm}{-0.5cm}   

  \begin{minted}[breaklines,fontsize=\footnotesize, linenos, bgcolor=bg, frame=lines]{c++}
void Quadtree::assignNode(int *bodies_per_node,
                          std::vector<std::vector<Node*> > &node_assignment,
                          Node &node)
{
    max_bodies_per_node = int(1.01 * root.nb_bodies/(nb_proc));

    if (node.nb_bodies > max_bodies_per_node && !node.is_leaf)
    {
	assignNode(bodies_per_node, node_assignment, *node.tr);
	assignNode(bodies_per_node, node_assignment, *node.tl);
	assignNode(bodies_per_node, node_assignment, *node.bl);
	assignNode(bodies_per_node, node_assignment, *node.br);
    }
    else if (!node.is_leaf)
    {
	int i = 0;
	while( ( (node.nb_bodies + *(bodies_per_node+i)) > max_bodies_per_node) && i < (nb_proc) )	
	    i++;
	if (i == nb_proc)
	{	  
	    assignNode(bodies_per_node, node_assignment, *node.tr);
	    assignNode(bodies_per_node, node_assignment, *node.tl);
	    assignNode(bodies_per_node, node_assignment, *node.bl);
	    assignNode(bodies_per_node, node_assignment, *node.br);
	}
	else
	{
	    node_assignment[i].push_back(&node);
	    *(bodies_per_node+i) += node.nb_bodies;
	}	
    }
    else if(node.is_leaf && node.contains_body)
    {
	node_assignment[nb_proc-1].push_back(&node);
	*(bodies_per_node+nb_proc-1) += 1;;
    }
}
  \end{minted}
\caption{Load balancing code. This is a tree-walk and each branch is either assigned to a compute node, or split into the sub-branches and then the process is repeated. This is a recursive function which allows the tree-walk to be performed easily.}
\label{fig:load-balancing}
  \end{adjustwidth}

\end{figure}
\subsection{Theoretical speedup}
The timing diagram (Figure \ref{fig:timing}) is nearly the same for both approaches. The only significant differences concerns the presence/absence of the quad-tree construction. The other differences which cannot be observed on the diagram, is the relative length of each step. For the brute-force approach, the computation step is multiple times longer than any other step, whereas for the quad-tree approach the difference in duration is not as large. The critical path is shown through the diagram as the black arrow. Since the synchronization between nodes is a blocking process, after each computation stage each node must wait for all the other to finish there calculations, this also emphasizes how import the load balancing is (see section \ref{sec:lb}). The timing diagram assumes all nodes are load balanced, the critical path after the initial calculation step can be any node; in practice, it will be the last node which finishes its own calculations.\\
From Figure \ref{fig:timing}, the theoretical speedup can be calculated. Since each node builds the same tree, this step can never be accelerated by parallelizing the code, as well as loading the data from a file. The communications must also be taken into account when estimating the theoretical speedup. Hence only the computation step can be accelerated, until it becomes insignificant compared to the other steps. The theoretical speedup can hence be derived knowing the following quantities:
\begin{itemize}
\item $t_{comm}$ as the time to send the positions, velocities and masses from master to all nodes
\item  $t_{building}$ as the time to build the quad-tree and the time to calculate the body assignment (see load balancing at section \ref{sec:lb})
\item $t_{comp}$ as the time to compute forces and update positions for all bodies on one node
\item $t_{sync}$ as the time to synchronize positions after one time step (this should be similar to $t_{comm}$ except that the function is an MPI\_Allgatherv instead of a MPI\_Bcast hence the times will be different)
\end{itemize}
Based on the measurements which are shown in Tables \ref{tab:qt:10e5} and \ref{tab:qt:10e6}, a constant time is assumed for both $t_{comm}$ and $t_{sync}$. For a few nodes, the computation time is significantly larger than the communication time, where as for large numbers of nodes the times become closer to constants. For the timing diagram drawn in Figure \ref{fig:timing}, hence 4 nodes and only one time step, the theoretical speedup formula is as follows:
\begin{align}
t_{serial}& = t_{building} + t_{comp}\\
t_{par} &= t_{comm} + t_{building } + \frac{t_{comp}}{4} + t_{sync}
\end{align}
\[S_{th} = \frac{t_{serial}}{t_{par}} = \frac{t_{building} + t_{comp}}{t_{comm} + t_{building } + \frac{t_{comp}}{4} + t_{sync}}\]
And this can be generalized to $k$ nodes: 
\[S_{th,k} = \frac{t_{building} + t_{comp}}{ t_{comm} + t_{building} + \frac{t_{comp}}{k} + t_{sync}}\]
This equation is valid under the assumption that the computation and quad-tree building times are significantly larger than the communication times, and that the nodes are perfectly load balanced. For the brute-force approach the same methodology is applied, except that the building time is not present (one can simply set $t_{building}=0$ in the formula for $S_{th,k}$).\\
The values which are used for the calculation of the theoretical speedups are based on Tables \ref{tab:bf:10e6}, \ref{tab:qt:10e5} and \ref{tab:qt:10e6}. A summary of these values is given in Table \ref{tab:summary:th_su}.
\begin{table}[H]
\centering
\begin{tabular}{l|cccccc}
Algorithm                    & \# bodies & $t_{serial}$ & $t_{comm}$ & $t_{building}$ & $t_{comp}$ & $t_{sync}$ \\
\hline\multirow{2}{*}{Brute-force} & $10^5$    & 453.94  & 0.0035 & -  & 453.81 & 0.0015\\
                             & $10^6$    & 44'875&  0.035 & -  & 44'874 & 0.025\\     
\multirow{2}{*}{Barnes-Hut}  & $10^5$    & 8.24 & 0.0038 & 0.2 & 8.23 & 0.003\\
                             & $10^6$    & 100 & 0.045 & 2.7 & 97.8 & 0.03\\
\end{tabular}
\caption{Values used for the calculation of the theoretical speedups.}
\label{tab:summary:th_su}
\end{table}

\begin{figure}[H]
\centering
\inputTikz{1}{../figures/timing_qt.tex}
\caption{Timing diagram for both solutions. The block of computations is repeated until the final time is reached. The circled characters next to the coloured boxes refer to Figure \ref{fig:flow_chart}, to emphasize that they are the same steps. The green step (C) is only present for Barnes-Hut algorithm, in the brute-force approach this step does no exist.}
\label{fig:timing}
\end{figure}

